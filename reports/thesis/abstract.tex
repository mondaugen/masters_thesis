This thesis explores a strategy of audio source separation that relies on a
classification of partials via their measured modulations. Perceptual studies
have shown that signals with common amplitude\hyp{} and
frequency\hyp{}modulation are heard as coming from the same source
\cite{mcadams1989segregation}, \cite{marin1991segregation}. To include these
modulations in a sum-of-sinusoids model a superlinear polynomial phase function
is adopted whose parameters are estimated using the Distribution Derivative
Method (DDM) \cite{betser2009sinusoidal}. For better estimation accuracy, a
window is designed that has lower side-lobes than the canonical Hann window but
that is also once-differentiable --- a requirement of the DDM. These estimated
parameters are used in a new partial tracking algorithm based on linear
programming. The resulting partials are classified using the clustering
technique of Gaussian mixture models \cite{friedman2001elements} on frequency-
and amplitude-modulation data. Principal components analysis is used to
emphasize the parameter on which it would be best to perform classification.
Once the partials have been classified into sources, the sources are synthesized
from the measured sinusoidal parameters.

The additional information provided by the DDM (namely the frequency and
log-amplitude slope) is incorporated into interpolating polynomials for the
phase and amplitude of sinusoids. The quality of different model-orders for
these polynomials is assessed on synthetic signals. The source separation system
is evaluated on both simulated data and on a mixture of real recordings of
percussive and plucked string instruments. In this latter case, it is shown that
using amplitude-modulation is a good criterion for separation when there is
little frequency-modulation.
