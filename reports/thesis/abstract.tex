Perceptual studies have shown that signals with common amplitude\hyp{} and
frequency\hyp{}modulation are heard as coming from the same source
\cite{mcadams1989segregation}, \cite{marin1991segregation}. This thesis examines
the classification of partials via their measured modulations. To include these
modulations in a sum-of-sinusoids model a superlinear polynomial phase function
is adopted whose parameters are estimated using the Distribution Derivative
Method (DDM) \cite{betser2009sinusoidal}. For better estimation accuracy, a
window is designed that has lower side-lobes than the canonical Hann window but
that is also once-differentiable --- a requirement of the DDM. These estimated
parameters are used in a new partial tracking algorithm based on linear
programming. The resulting partials are classified using the clustering
technique of Gaussian mixture models \cite{friedman2001elements} on frequency-
and amplitude-modulation data. Once the partials have been classified into
sources, the sources are synthesized from the measured sinusoidal parameters.
The additional information provided by the DDM is incorporated into
interpolating polynomials for the phase and amplitude of sinusoids. The quality
of different model-orders for these polynomials is assessed on synthetic
signals. The system is evaluated on both simulated data and on a mixture of real
recordings of musical instruments.
