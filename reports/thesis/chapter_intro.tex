\chapter{Introduction}
In signal processing, a common task is the separation of a signal with known
deterministic or statistical characteristics from another. This task has been
well studied \cite{kay1993fundamentals} \cite{hayes2009statistical}
\cite{poor2013introduction} and works well for problems of digital communication
or object detection where these characteristics are well known. In digital
communication, the signals and techniques to transmit them are often optimized
by the designer to make them robust to corruption or interference. The designers
of vehicles usually design them to be predictable and reliable and so their
positions in time will reflect this. In this thesis we tacle a more difficult
problem, that of the separation of a mixture of acoustic signals. The nature of
these signals is different in that their design criteria are either
mostly unknown or fundamentally different. For example, musical instruments are
designed to have desirable acoustic properties which are generally subjective.
As an example of one of the complications, the choirs of the orchestra are sets
of instruments actually designed to blend well; to sound as one instrument.
Ironically, it is this criterion that we use to guide the source separation
techniques described in the following.

