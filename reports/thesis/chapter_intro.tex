\chapter{Introduction}
In signal processing, a common task is the separation of a signal with known
deterministic or statistical characteristics from another. This task has been
well studied \cite{kay1993fundamentals} \cite{hayes2009statistical}
\cite{poor2013introduction} and works well for problems of digital communication
or object detection where these characteristics are well known. In digital
communication, the signals and techniques to transmit them are often optimized
by the designer to make them robust to corruption or interference. The designers
of vehicles usually design them to be predictable and reliable and so their
positions in time will reflect this. In this thesis we tacle a more difficult
problem, that of the separation of a mixture of acoustic signals. The nature of
these signals is different in that their design criteria are either
mostly unknown or fundamentally different. For example, musical instruments are
designed to have desirable acoustic properties which are generally subjective.
As an example of one of the complications, the choirs of the orchestra are sets
of instruments actually designed to blend well; to sound as one instrument.
Ironically, it is this criterion that we use to guide the source separation
techniques described in the following.

\section{Notation}

\subsection{Vectors and matricies}

While scalars are typeset normally --- $x$ is an example of a scalar --- vectors
and matrices are typeset in a boldface font, with matrices written with a
capital letter, e.g., $\boldsymbol{x}$ is a vector and $\boldsymbol{X}$ a
matrix. If a number is written instead of a symbol, we mean a vector all of that
number, e.g., $\boldsymbol{1}$ is the vector of all $1$s, $\boldsymbol{0}$ the
vector of all $0$s. The $i$th entry of a vector $\boldsymbol{x}$ is written
$x_{i}$ and the entry in the $i$th row and $j$th column of a matrix $X$ is
written $X_{i,j}$.  Both are scalars and therefore typeset normally. Sometimes
we might find it convenient to extract a column vector or row vector from the
matrix $\boldsymbol{X}$. We write $\boldsymbol{x}_{i,:}$ to extract all columns
from the $i$th row and and $\boldsymbol{x}_{:,j}$ to extract all rows from the
$j$th column. These are the $i$th row vector and $j$th column vector
respectively. The orientation of a vector will be clear from the context, but in
general $\boldsymbol{x}$ is a column vector while $\boldsymbol{y}_{i,:}$ and
$\boldsymbol{x}^{T}$ are row vectors.

\subsection{Operators}

\subsubsection{Inner product}

We will be dealing with objects in vector spaces. The operator $\left\langle
x, y \right\rangle$ takes two objects in a vector space $V$, $x,y \in V$ and maps
them to an element $k \in \mathbb{K}$ of a field $\mathbb{K}$.
For this thesis, the field will always be the field of real numbers
$\mathbb{R}$ or complex numbers $\mathbb{C}$. The vector space can be the set of
vectors of $N$ elements in $\mathbb{R}^{N}$ or $\mathbb{C}^{N}$, in which case
the inner product is defined, for $\boldsymbol{x},\boldsymbol{y} \in
\mathbb{K}^{N}$, $k \in \mathbb{K}$
\[
    \left\langle  \boldsymbol{x}, \boldsymbol{y} \right\rangle =
\boldsymbol{x}^{T} \boldsymbol{y} = k
\]
The inner product is also defined on
the vector space of functions $\Phi$
mapping from a set $S$ to a field $\mathbb{K}$, $\Phi : \forall f \text{ s.t. } f(s) = k, s \in S,k \in
\mathbb{K} $ in which case the inner product on $g,f \in \Phi$ is
defined as
\[
    \left\langle  g, f \right\rangle =
\int_{-\infty}^{\infty} g(x) \overline{f(x)} dx
\]
and $\overline{a}$ gives the complex conjugate of $a$.

\subsubsection{General outer operators}

The outer operator $\cdotp \otimes_{\mathcal{O}} \cdotp$ will only be defined for
vectors in this thesis. It operates on the two vectors $\boldsymbol{x},\boldsymbol{y} \in
\mathbb{K}^{N}$ and is defined as
\[
    \boldsymbol{x} \otimes_{\mathcal{O}} \boldsymbol{y} = \boldsymbol{W}
\] where the $i$th row and $j$th column of $\boldsymbol{W}$ are
\[
    w_{i,j} = \mathcal{O} (x_{i},y_{j})
\]
Canonically, the operator $\mathcal{O}$ is multiplication in which case
\[
    \boldsymbol{x} \otimes_{\times} \boldsymbol{y} = \boldsymbol{W}
\] where the $i$th row and $j$th column of $W$ are
\[
    W_{i,j} = x_{i} y_{j}
\]
but $\mathcal{O}$ can be defined arbitrarily as any function taking two inputs a
returning a single output. The general outer product is also known as the
\textit{Kronecker product}.

\subsubsection{Point-wise operators}

If an operator on matrices $\circ$ is written with a period preceding it, i.e.,
$.\circ$ it
means perform that operation on each element individually. Some examples follow.

For matrix $\boldsymbol{X} \in \mathbb{K}^{M,N}$ and $p \in \mathbb{K}$
\[
    \boldsymbol{X}^{.p} = \boldsymbol{W}
\] where
\[
    W_{i,j}=X_{i,j}^{p}
\]
For matricies $\boldsymbol{X},\boldsymbol{Y} \in \mathbb{K}^{M,N}$
\[
    \boldsymbol{X}.\boldsymbol{Y} = \boldsymbol{W}
\] where
\[
    W_{i,j}=X_{i,j}Y_{i,j}
\]
(constrast these with canonical matrix multiplication).

\subsection{Random variables}

Many authors denote random variables with a normally typeset uppercase letter.
We will use this convention only when convenient, but will always state
explicitly that a certain variable is random.

\subsection{Complex numbers}

A complex number $z \in \mathbb{C}$ can be described in cartesian notation as
\[
    z = a + jb, a,b \in \mathbb{R}
\]
or in polar notation as
\[
    z = \alpha \exp(j\omega), \alpha,\omega \in \mathbb{R}
\]
where $j = \sqrt{-1}$. $j$ is also often used to denote an index variable. It
will be clear from the context when the imaginary number is meant and when the
index.

\subsection{Logarithms}

The logarithm base-$e$\footnote{$e$ is Euler's constant.} of $x$ is written
$\log(x)$. The logarithm of any other base $b$ will be denoted as such:
$\log_{b}(x)$.
