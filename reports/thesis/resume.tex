Dans ce mémoire nous explorons une stratégie de séparation de sources sonores
s’appuyant sur une classification de partiels selon leurs modulations observées.
Des études perceptives ont montré que des signaux dont les modulations
d’amplitude et de fréquence sont communes sont perçus comme provenant d’une même
source \cite{mcadams1989segregation}, \cite{marin1991segregation}. Afin
d’inclure ces modulations dans le modèle de synthèse additif, la phase est
représentée par une fonction polynomiale non-linéaire dont les paramètres sont
estimés par la méthode de distribution dérivée (Distribution Derivative Method -
DDM) \cite{betser2009sinusoidal}. Afin d’améliorer la qualité d’estimation, nous
avons conçu une fenêtre dont la résolution dynamique est meilleure que celle de
la fenêtre canonique de Hann, tout en étant dérivable sur tout son domaine,
propriété requise par la technique DDM. Les paramètres ainsi estimés sont
utilisés par un nouvel algorithme de suivi de partiels fondé sur le principe
d’optimisation par programmation linéaire. Les partiels trouvés sont alors
classifiés en différentes sources par une technique de mélange gaussiens
appliquées aux paramètres de modulation de fréquence et d’amplitude. Au
préalable une analyse en composantes principales (PCA) est utilisée afin de
faire ressortir les paramètres les mieux appropriés pour la classification.  Une
fois les partiels regroupés et classifiés en sources, celles-ci sont
synthétisées en fonctions des paramètres associés aux trajets de partiels.

Plus précisément, les informations additionnelles fournies par la DDM (dérivée
de la fréquence et de la log-amplitude) sont prises en compte selon plusieurs
stratégies impliquant des polynômes de reconstruction de phase et d’amplitude
d’ordres différents. La qualité des signaux re-synthétisés est alors évaluée
pour chacune de ses stratégies. 

Enfin, ce système de séparation de sources est testé sur un mélange de signaux
synthétiques puis sur un mélange de signaux instrumentaux réels de percussion et
de corde pincée. Dans ce dernier cas, nous montrons que la prise en compte de la
modulation d’amplitude aide à la classification en l’absence de modulation en
fréquence.
