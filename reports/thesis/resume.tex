Ce mémoire va explorer une stratégie pour séparer les sons en mixture en
utilisant une classification de partiels par leurs modulations. Les études sur
la perception de son montrent que les signaux avec les modulations d'amplitude
et fréquence en commun se ressemble \cite{mcadams1989segregation},
\cite{marin1991segregation}. À inclure ces modulations dans le modèle de son
``somme des sinusoïdes'', on adopte une modèle de polynôme superlinéaire a
décrire leurs phases, dont l'estimation des paramètres est par le ``Distribution
Derivative Method'' (DDM). Pour améliorer l'estimation, on propose une fenêtre
avec les propriétés supérieur à la fenêtre canonique Hann, mais que admet une
dérivée sur tout sa domaine, ce qui est requis par le DDM. Un nouveau algorithme
pour la poursuite des partiels basé sur la programmation linéaire utilise ces
paramètres estimés.  Une technique de classification par les mixtures gaussienne
se sert des données de modulation en fréquence et d'amplitude des partiels
trouvés à chercher les sources originales. Ces données son traitées par un
analyse des composants principaux pour trouver le meilleur paramètre pour la
classification. En suivant, les sources sont synthétisées avec leurs paramètres
estimés. En cherchant les polynômes interpolant de la phase et de l'amplitude,
on inclus l'information additionnelle donnée par le DDM et évalue la qualité de
la synthèse avec les polynômes d'ordres différents sur les signaux synthétiques.
Ce système pour séparer les sons est évalué sur les données simulées et sur une
mixture des enregistrements des instruments de percussion. Ce dernier montre que
l'utilisation de la modulation d'amplitude aide à la classification en l'absence
de la modulation en fréquence.
