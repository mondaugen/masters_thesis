\chapter{Methodology}
Perceptual studies have shown that sounds modulated synchronously in amplitude
or frequency are heard as one sound, whereas asynchronously modulated sounds are
heard as distinct \cite{mcadams1989segregation} \cite{marin1991segregation}.
Here we define the modulation of parameters $\theta_i$ and $\theta_j$ as being
synchronous if they are given as functions of time, $\theta_i=f_i(t)$ and
$\theta_j=f_j(t)$ and there is an affine transform $\mathscr{A}$ such that
$\mathscr{A}\{f_i\}(t) \approx A f_j(t) + B$ where $A$ and $B$ are constants
that do not vary with time (at least for the time that we observe the signal).
If we can accurately measure these parameters and they are typical of the sounds
we are trying to separate, then we can design techniques to reliably separate
these sounds from acoustic mixtures. This involves picking those parameters
classified as belonging to the same sound, discarding the rest, and
resynthesizing from these parameters. The task of audio source separation
therefore comprises the following tasks:

\begin{itemize}
    \item
        Decide on a signal model for the sound of interest, with parameters that
        can be estimated and that are similar for similar sounds.
    \item
        Estimate the signal parameters.
    \item
        Classify the parameters and group them as sets of parameters coming from
        the same source: the sound of interest.
    \item
        Choose a group of parameters and synthesize the separated signal from
        them.
\end{itemize}

This thesis describes various approaches to the above tasks and utilises some of
them in source separation experiments.

