\chapter{Conclusion\label{chap:conclusion}}

Here we summarize the results of our work, highlight our contributions to the
audio source separation problem, and suggest some possible extensions to the
techniques presented in this thesis.

\section{Results}

\subsection{Quality of polynomial models for analysis and synthesis}

In Chapter~\ref{chap:exphsmodel} we explored various orders of polynomial amplitude and
phase models for analysis and synthesis and assessed their quality. A
quadratic estimation of the phase and log-amplitude in the analysis step
improved the synthesis quality over the constant amplitude and linear phase
model in all cases considered. The use of a quintic phase
polynomial in the synthesis step better approximated a signal with exponential
phase. However, the use of the quintic phase and amplitude polynomial did not
necessarily improve the synthesis of a signal with cubic phase and quartic
amplitude. Nevertheless, the additional information provided by the DDM in the
analysis stage was shown useful in postulating interpolating polynomials at the
synthesis stage.

\subsection{The use of amplitude-modulation in audio source separation}

Chapters~\ref{chap:amfmsep}~and~\ref{chap:decaysep} explore the contribution of
amplitude-modulation to the audio source separation problem on simulated data.
In Chapter~\ref{chap:amfmsep} we consider both the amplitude- and
frequency-modulation when classifying into sources. We adaptively selected the
best variable on which to classify using principal components analysis and found
that amplitude-modulation aided in the case of little frequency modulation. A
post-processing step was required to extract the original sources. This step
extracted sources based on the smoothness between analysis frames of the
amplitude and frequency measurements.  Only the post-processing encouraging
smoothness in frequency gave encouraging results --- smoothing in amplitude
failed due to the rapidly varying amplitude in the transient regions of the
sounds. In Chapter~\ref{chap:decaysep} we apply our source separation technique
to a recording of a percussive and a plucked string instrument. These
instruments exhibit virtually no frequency-modulation. We showed, as in
Chapter~\ref{chap:amfmsep}, that it is
possible to carry out source separation when the difference in
amplitude-modulation of the two sources is sufficient. 

\section{Contributions}

\subsection{Design of continuous windows with lower side-lobes}

The DDM requires the use of windows that are once differentiable. The canonical
window with this property is the Hann window, however the Blackman-Harris window
has better properties in terms of its out-of-band signal rejection. In
Chapter~\ref{chap:sigmod}, starting
from the description of a Blackman-Harris window, we used optimization
techniques to search for a window close to the Blackman-Harris, but whose
end-points are 0, thus allowing differentiation everywhere after windowing with
a rectangular window (necessary to make the window have finite support).
Informally, we found this to improve the estimation accuracy of the DDM.

\subsection{Partial tracking using linear programming}

The original peak-matching algorithm of McAulay and Quatieri
\cite{mcaulay1986speech} searches for partials through a lattice of spectral
analysis data by considering optimal peak-matches between two adjacent frames.
In Chapter~\ref{chap:partialtracking} we show that this can be generalized to
search for $L$ paths between an arbitrary number of a frames in a lattice, and
that it is a greedy algorithm --- one that gives the shortest possible path
without considering the lengths of the other paths in the solution. Also, it is
shown that the complexity of this algorithm is exponential in the number of
frames and therefore impractical to apply to even moderately sized problems. A
linear program to search for the $L$ paths through a lattice with shortest total
cost is introduced. We show that this algorithm is computationally tractable and
that it outperforms the McAulay-Quatieri method on signals with a low SNR.

\section{Future extensions}

\subsection{Continuous analysis windows}

It remains to be seen systematically in what situations the continuous
Blackman-Harris window improves the estimation accuracy of the DDM. For this,
DDM-based analysis using both the Hann and continuous Blackman-Harris windows
should be performed on mixtures of synthetic signals and the accuracy of
analysis compared. Further improvements to the window design could be
investigated, such as finding a continuous Blackman-Harris window with good
trade-off between main-lobe width and side-lobe height.

\subsection{Partial tracking in an optimization framework}

The reinterpretation of partial tracking as an optimization problem allows the
application of techniques from mathematical programming, a field with
a rich body of knowledge \cite{boyd2004convex}. As discussed in
Chapter~\ref{chap:partialtracking}, regularization could be used to encourage
less ``jagged'' partial trajectories. The merits of partial tracking via
linear programming should be assessed on real music and speech signals.

\subsection{Signal modeling with nonlinear amplitude and phase polynomials}

A non-stationary description of signals could make it possible to describe
signals using fewer analysis frames as the signal can be more accurately modeled
between analysis points. As mentioned in \cite{betser2009sinusoidal}, it should
be investigated whether or not such a model could contribute the field of audio
coding. Specifically, compression ratios should be compared between analyses
using a lower order model, but with more analysis frames, and those using higher order
models but fewer analysis frames. Possible improvements to applications such as
the time-stretching of transients and the extrapolation of missing signals
(inpainting) should also be considered.
