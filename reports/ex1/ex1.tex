\documentclass[letter,10pt]{report}
\author{Nicholas Esterer}
\title{Experiment 1: Partial grouping in one frame}
\begin{document}
\maketitle
\tableofcontents
\section{Introduction}
To evaluate whether the grouping of partials with common AM and FM parameters is
plausible, we synthesize a set of parameters and test by corrupting the
parameters with noise and adding spurious sets of parameters (that should no
belong to any sources).
\section{Methodology}
We synthesize theoretical sets of parameters as described above. On each frame
of analysis data, i.e., for parameters belonging to the same time instant, we
consider each data point as a multi-dimensional random variable. With these
random variables, we compute principle components in order to produce a variable
with maximum variance. This variable is classified using a clustering algorithm
and we evalutate the results. A summary follows:
\begin{itemize}
\item 
Parameters are synthesized from a theoretical mixture of AM and FM sinusoids.
Spurious data are added to these parameters.
\item
Principle components analysis is carried out on the parameters happening at
one time instance.
\item
A histogram is made of the 1st principle components. Values sharing a bin with
too few other values are discarded to remove spurious data points.
\item
Initial means and standard deviations for the Gaussian mixture models are made
by dividing the histogram into equal parts by area and choosing the centres of
these parts.
\item
The EM algorithm for Gaussian mixture models is carried out to classify the
sources.
\end{itemize}
